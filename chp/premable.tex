
\par
% \centering
\textbf{前言}
\par


初衷:
在做图像处理与医学成像的过程中,经常与遇到各种各样的医学图像格式,包括自然图像格式jpg, png, bmp, tiff等,但更多的涉及到医学成像领域的,以及图像后处理领域的。
医学成像领域的主要包括CT, MRI, PET, US, NI, DR,X-machine等,这里涉及到不同的医学成像设备制造商,他们各自的格式也不一样。比如SIMENS的.dat格式,GE的raw格式, PHILLIP的格式,以及一些专用的医学图像格式,比如nii格式,dcm格式,mnc格式等,还有一些软件里面涉及到的格式,比如matlab里面的mat数据格式,csv格式等,对于医学图像处理软件来说,图像数据的读写和显示是首先要解决的事情,只有先解决好图像文件的读写和可视化问题,后续的重建算法,图像后处理算法方可实现。


第一章主要概述各种不同的图像格式,以及它们的相关文档,官方网站,以及相关的读写及可视化工具的介绍。


第二章详细介绍自然图像格式,比如bmp, jpg, png, tiff等格式的技术标准,以及在各种不同的编程语言环境下如何实现读写以及可视化。

第三章详细介绍医学成像过程中的各种不同的数据格式,主要从不同的成像模态来区分,同时介绍一下成像原理以及不同的数据文件格式中物理量的含义。主要涉及的医学成像模态主要包括MRI,CT,DR, PET, US, NI, SPECT等。其中MRI数据是k空间数据,CT数据是正弦图sinograms数据, PET是
